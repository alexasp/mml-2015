\usepackage{setspace}
\usepackage{graphicx}
\usepackage{grffile}
\usepackage{amssymb}
\usepackage{mathrsfs}
\usepackage{amsthm}
\usepackage{amsmath}
\usepackage{verbatim}
\usepackage{color}
\usepackage[]{algorithm2e}
\usepackage[Lenny]{fncychap}
\usepackage[pdftex,bookmarks=true]{hyperref}
\usepackage[pdftex]{hyperref}
\hypersetup{
    colorlinks,%
    citecolor=black,%
    filecolor=black,%
    linkcolor=black,%
    urlcolor=black
}


%
\usepackage[font=small,labelfont=bf]{caption}
\usepackage{fancyhdr}
\usepackage{times}
\usepackage{csquotes}
%\usepackage[intoc]{nomencl}
%\renewcommand{\nomname}{List of Abbreviations}
%\makenomenclature
\usepackage[square]{natbib}
\usepackage{float}
%\usepackage[breaklinks,hidelinks]{hyperref}
\usepackage{tabulary}
\usepackage{array}
\usepackage{hyperref}
%\usepackage{biblatex}
%\floatstyle{boxed} 
%\restylefloat{figure}

\usepackage{xargs}                      % Use more than one optional parameter in a new commands
\usepackage[pdftex,dvipsnames]{xcolor}  % Coloured text etc.
% 
\usepackage[colorinlistoftodos,prependcaption,textsize=tiny]{todonotes}
\newcommandx{\unsure}[2][1=]{\todo[linecolor=red,backgroundcolor=red!25,bordercolor=red,#1]{#2}}
\newcommandx{\change}[2][1=]{\todo[linecolor=blue,backgroundcolor=blue!25,bordercolor=blue,#1]{#2}}
\newcommandx{\info}[2][1=]{\todo[linecolor=OliveGreen,backgroundcolor=OliveGreen!25,bordercolor=OliveGreen,#1]{#2}}
\newcommandx{\improvement}[2][1=]{\todo[linecolor=Plum,backgroundcolor=Plum!25,bordercolor=Plum,#1]{#2}}
\newcommandx{\thiswillnotshow}[2][1=]{\todo[disable,#1]{#2}}


\usepackage[number=none]{glossary}
\makeglossary
\newglossarytype[abr]{abbr}{abt}{abl}
\newglossarytype[alg]{acronyms}{acr}{acn}
\newcommand{\abbrname}{Abbreviations} 
\newcommand{\shortabbrname}{Abbreviations}
%\makeabbr
\newcommand{\HRule}{\rule{\linewidth}{0.5mm}}
%\linespread{1.2}



\renewcommand*\contentsname{Table of Contents}

\theoremstyle{definition}
\newtheorem{definition}{Definition}

\pagestyle{fancy}
\fancyhf{}
\renewcommand{\chaptermark}[1]{\markboth{\chaptername\ \thechapter.\ #1}{}}
\renewcommand{\sectionmark}[1]{\markright{\thesection\ #1}}
\renewcommand{\headrulewidth}{0.1ex}
\renewcommand{\footrulewidth}{0.1ex}
\fancypagestyle{plain}{\fancyhf{}\fancyfoot[LE,RO]{\thepage}\renewcommand{\headrulewidth}{0ex}}