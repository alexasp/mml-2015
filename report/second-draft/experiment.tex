%===================================== CHAP 4 =================================

\chapter{Experiment}

\section{Overview}

As presented in Section \ref{sec:problem_statement}, we wanted to do test an architecture that allows fully decentralized machine learning that maintains the privacy of the participants. 

We consider a setting with $N$ peers that each have a local data set. These data sets are assumed to be independently sampled for each peer, but may be sampled from the same distribution. When the system initializes, each peer trains a logistic regression model on its local dataset. The data set and the trained model is private and should only be known by its owner.

After the initialization phase, the aggregation phase begins. In this phase, the aggregation mechanism described in Section \ref{sec:Sensitivity_of_LogReg} is applied one or more times, using the private model held by each peer. The mechanism is not applied to all peers at the same time. Instead, subsets of peers are selected randomly to form aggregation groups, each group producing a single aggregate model. Many such groups can be formed, and the group size can vary from including all the available peers to including just a single peer. In our experiments, we specify a constant group size which is used until the end of the experiment.

While the output of each mechanism application is an average of the input models, produced in way that guarantees differential privacy, the computation itself must be done in a central manner. This is possible to do securely using the protocol detailed by Pathak et al., which uses homomorphic encryption to compute the model aggregate without allowing the any of the participants to know the original, private model of another participant\cite{pathak2010diffprivhomo}. Since this protocol requires some central computation, one of the peers is chosen at random to be the curator, responsible for acting as the central party described in the solution by Pathak et al. The other peers in the group will submit the necessary information to this curator, including their private model, in an encrypted fashion. Once the peer acting as curator has received a model from all participants, it computes the average model, adds noise sufficient to guarantee $\epsilon$-differential privacy and publishes the final result. The target of this publish step can vary. In our experiments we have tested one version that publishes a model to all available peers and one that only publishes the model to the peers in the group that helped create it.

Each peer holds a privacy budget, as discussed in Section \ref{section:privacy_budget}, that limits how many times it can be involved in a mechanism application. If the budget of a peer is depleted, it is no longer a candidate for the randomly formed aggregation groups. When there no longer is enough peers to form a group with the size specified for a particular experiment, the experiment terminates.

\subsection{Limitations of current implementation}

Certain parts of this system is not implemented in this project, and are replaced by black-box substitutes that simulate the required behavior. We have only done this for components that are already described and tested in other work. 

The protocol created by Pathak et al. for computing aggregates securely was not implemented. In our implementation models are sent unencrypted to the peer acting as curator. While this part would need to be replaced with a full implementation of the approach by Pathak et al. the output returned by the curator is exactly the same, using the computation in Equation \ref{eq:parametric_aggregation}. 

Finally, the selection of random groups is done in a non-scalable manner. A centralized actor that has full knowledge of all participating peers randomly selects groups of these peers and sends a message to each peer with the list of participants. This should be replaced by a decentralized method for the system to be scalable. How we intend to do this is discussed further in Section \ref{sec:Future Work} on Future Work.

\section{Dataset}
This section will introduce the dataset(s) used. What features it contains, what we try to learn/classify, and why we chose to use it.

\subsection{Spambase} \label{sec:spambase}
The Spambase dataset \cite{spambase1999data} was used as a baseline training set. This dataset is publicly available from the UCI machine learning directory, and contains 57 input attributes of continuous format which serves as input features for spam detection and 1 target attribute in discrete format which represents the class.

We chose this dataset as it is a popular dataset to analyze the performance of binary classifiers, so that we could compare the results of other logistic regression classifiers against our own. While this dataset might not seem like the ideal choice for testing a differentially private classifier due to its lack of personal information, we argue that it still fits well for the purpose of demonstration. In a spam-classifying system based on our distributed model, a logistic regression model can be built by training it locally in each user's personal mail folder and then aggregated into an ensemble. That way you can build a diverse spam-classifier without the users having to give up their personal email to a centralized database.     

Before we could use the dataset, we needed to use normalization to scale the data to 0-1 range. This is due to the proof in Chaudhuri paper which states the assumption $|X_i|< 1$. The scaling was  based on the formula
\begin{eqnarray}
	X_{norm} = \frac{X-X_{min}}{X_{max} - X_{min}}
\end{eqnarray}

We appended a feature with constant value 1.0 to all data records, to act as the intercept or bias term. 

\subsection{Australian Credit Approval}
Another dataset we used, was the Australian Credit Approval (Statlog), which concerns the approval or disapproval of credit card applications. It is publicly available from the UCI machine learning directory. This is a much smaller set of data, with only 690 samples spread over 14 attributes. The data is useful for binary classification as it contains a good mix of attributes: continuous, nominal with small numbers of values, and nominal with larger numbers of values.

We mostly used this dataset to confirm and double-check the conclusions we drew from the Spambase dataset, as it contains too few data records for it to be 100\% ideal for our use. Due to the sparsity of data we had to scale down the amount of peers in each experiment, so that each peer could get enough data to create a decent local classifier. We chose to keep on using this data as one of the motivating examples in Dwork's book\cite{dwork2013algorithmic} of differential privacy, is to protect data holders from insurance and credit card companies. 

This dataset was preprocessed in the same manner as was described in Section \ref{sec:spambase}.

\section{Parameter tuning}
\label{sec:parameter_tuning}
Number of peers $P$ specifies how many different peers participate in the experiment, and necessarily the number of partitions of the training data sets. The training is divided into $P$ parts of equal size.

\todo[inline]{Rationalize why we have included 1 in the 10-inteval experiments - it is because 1 is a very interesting edge case. Should also talk about the significance of group size 1  in analysis.}

Aggregate models are created from local models at each peer through an aggregation process that is performed one or more times with subsets of peers. The parameter $g$ specifies how many peers will participate in a single model aggregation. Since each peer has a unique subset of data, this parameter determines how many partitions of the training set contribute to the published aggregate models. These data partitions do not contribute directly, but indirectly through the aggregation of models trained locally on each partition.

Each peer trains a local logistic regression classifier on its data partition. This requires selection of a learning rate $\alpha$, a regularization constant $\lambda$ and a maximum number of iterations of gradient descent $I$. The learning rate is sensitive to the size of the local training set\cite{wilson20013learningrate}, and should be tuned individually by each peer. We did this by running 3-fold cross validation when each peer fits its local model to identify the best $\alpha$ in the range $[-7, 0]$. 3-fold cross validation was chosen because of both project computer time constraints and experimental data constraints. Each experiment in its entirety is tested with 10-fold cross validation, so it was necessary to reduce local model training time in order to run in a reasonably short time on a single computer. The data constraints is a part of the domain we want to explore. When the amount of data is very small, 3-fold cross validation offers a balance between parameter search reliability and validation set sizes.

In usual data mining applications the regularization $\lambda$ would be tuned in this manner as well, but the sensitivity of the aggregation mechanism depends on $\lambda$, as seen in Equation \ref{eq:aggregated_logistic_sensitivity}. This means that the peers will have to communicate to either agree on a regularization level or to determine the smallest regularization constant to identify the worst case noise level. In our experiments we chose a global regularization level, which was used by all peers. We identified the best $\lambda$ by testing a coarse grid of powers of 2 whenever we changed the per-peer number of training samples.

\todo{Talk about the problem of tuning regularization. We are essentially doing a global selection of regularization. This could be difficult. Better to communicate the minimum, perhaps? But then, which regularization should peers pick locally? The highest one that stil has good performance?}

The privacy parameter $\epsilon$ determines the level of privacy for each data partition. Note that this parameter does not apply to the original training set as a whole - each peer has its own private database, which is protected by $\epsilon$-differential privacy. 

Finally, the parameter $\epsilon$ can be divided across several applications of the aggregation mechanism, as described in Section \ref{section:privacy_budget}. This was achieved with a per-aggregation parameter $\epsilon_i$. Each data partition can participate the aggregation mechanism $n$ times, where $n\epsilon_A \leq \epsilon$.


\section{Validation}

The test sets set aside could not be used when tuning and evaluating system hyperparameters. In order to explore the effects of the various hyperparameters we used cross validation with number of folds $n=10$. For a given combination of hyperparameters, performance metrics were measured as their average across ten repetitions. In repetition $i$, data fold $i$ was used as validation set and the remaining $n-1$ data folds were combined to form the test set.

\section{Algorithm}

This section explain the logistic regression algorithm, how it is commonly used, and what modifications are needed when used in a distributed setting. Explanation on how it is used in a differentially private manner is explained in the architecture section. 


\subsection{Application of Aggregation Mechanism}


The central element in our experiments is the aggregation mechanism $A$, which takes a set of models. This mechanism is given in Algorithm \ref{alg:privacy_mechanism}. As presented in Section \ref{sec:Sensitivity_of_LogReg}, the sensitivity of logistic regression depends on the sizes of the data sets used to train the models. Specifically, the mechanism needs to know the size of the smallest training set in order to guarantee differential privacy. It is important to note that the method we are testing assumes honest-but-curious participants, as assumed by Pathak et al\cite{pathak2010diffprivhomo}.

\begin{algorithm}[H]
	\label{alg:privacy_mechanism}
		\KwIn{$\epsilon$ - privacy parameter
			$\boldmath{M}$ - set of models trained by participating peers\;
			$\boldmath{N}$ - set of peer training set sizes\;
			$\lambda$ - regularization level used when training each model in $\boldmath{M}$\;
			}
		\KwOut{Perturbed aggregate of the models in $\boldmath{M}$}
		$n_{min} \leftarrow min(\boldmath{N})$\;
		$\eta \leftarrow Laplace(0; \frac{2}{n_{min}\epsilon\lambda})$\;
		$model_{agg} \leftarrow 1/K\sum_{j=1}^{|\boldmath{N}|} \boldmath{w_j} + \boldmath{\eta}$\;
		\KwRet{$model_{agg}$}
		
\caption{$\epsilon$-differentially private aggregation mechanism}
\end{algorithm}

\begin{algorithm}[H]
	\KwIn{$P$ - the set of peers\;
		$\epsilon$ - privacy parameter\; 
		$\epsilon_A$ - privacy level of a mechanism application\;
		$A$ - the $\epsilon_A$-differentially private aggregation mechanism\;
		$group\_size$ - number of peers in a single mechanism application}
	\For{peer $\in$ P}{
		$budget_{peer} \leftarrow \epsilon$\;
	}
	\While{$|P| \geq group\_size$}{
		$group \leftarrow randomSample(P, group\_size)$\;
		$model_{agg} \leftarrow A(group)$\;
		\For{peer $\in$ group}{
			$budget_{peer} \leftarrow budget_{peer} - \epsilon_A$\;
			\If{$budget_{peer} < \epsilon_A$}{
				$P \leftarrow P \smallsetminus peer$\;
			}
		}
		$publish(P, model_{agg})$
}
\caption{Distributed training process}
\end{algorithm}


\subsection{Propagation Of Published Models} \label{sec:PropagationPubModel}

Originally in our system, aggregated models were only propagated to the peers that had participated in creating that model, as can be seen in Figure \ref{fig:peerAggregationFigure}. What resulted from this, especially when epsilon was set to a low amount such as 0.1 or lower, was that the high amount of noise made the classifiers have a big standard deviation on their mean classification rate. What this meant was that while the classifiers could be very accurate in some peers, classifying up towards 90\% accuracy, it could also be significantly worse in other peers. 

We theorized that we could improve the ensemble classifier in each peer if we could propagate the aggregated models to all the peers in the network, instead of just those who had participated in making them. Our hypotheses was that this would lead to more stable classifiers with lower standard deviation, due to a smoothing effect in having more models in the ensemble classifier in each peer. This is basically the same idea as bootstrap aggregating, or bagging, which has been proven to lead to improvements in unstable procedures\cite{breiman1996bagging}. 
\todo{Definitely talk more about the bagging effect, either here or in the analysis section}

For this reason, we decided to run experiments to compare the different possible model. In all cases, the published models will have been perturbed with Laplacian noise to give $\epsilon$-differential privacy. In the group publication setting, only the peers that join together to produce a perturbed model will receive the final result. In the full publication setting, all peers active in the network will receive all perturbed models. 

Note that there is no selection or pruning of the ensemble classifier owned by each peer. If a peer receives a model, it will blindly add it to the ensemble. This means each peers ensemble model will grow much faster in the full publishing setting, and they will all contain essentially the same models, the only exception being the unperturbed model produced by the peer locally. We anticipated that this would lead to a reduction in ensemble model accuracy variance.


\section{Architecture}

We designed a distributed system using the JADE framework. The core component in this system is a PeerAgent, which represents a participant in the distributed learning setting. This agent contains what would be the local data of a person using some application. In the remaining sections, whenever we say "peer" we are refering to the PeerAgent described here, holding a local data set and with means of communcating with other PeerAgent instances.

To form aggregate models it is necessary to select groups of peers to create each model. In our experiment, this is implemented with a singleton agent we named the GroupAgent. This agent draws random subset. The size and number of groups formed is given by the parameters selected at the beginning of the experiment, as specified in Section \ref{sec:parameter_tuning}. It is this agent that is responsible for keeping track of 
%talk about how the stuff explained in Basic theory like AMS plays into our setup

%when we are talking about peers, we are refering to instance of the PeerAgent


 \missingfigure[figwidth=6cm]{Figure explaining our framework}


\cleardoublepage