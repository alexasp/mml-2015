%===================================== CHAP 2 =================================

\chapter{Background and Motivation}

In this section we will first explain some basic concepts and expressions that are used in the privacy context such anonymization operations and Personally Identifiable Information(PII). Then we will have a look at some classic examples of failure to preserve privacy when data publishing and how these attacks motivated us to choose our topic for this thesis.

\section{Concepts and Expressions}

\section{Privacy Breaches}
In the recent years there have been many failures in privacy preserving data publishing. Many companies have been faced with a PR disaster after releasing data about their customers thinking them being anonymized, only to have people de-anonymize their data and breaching the privacy of the datasets' participants. In this section we will have a look at some of these privacy failures.

\subsection{Netflix Prize Competition}
Netflix, the world's largest online movie streaming website, decided in 2006 to crowdsource a new movie suggestion algorithm and offered a cash prize of 1 million dollar for the most efficient algorithm. To help the research, they released 100 million supposedly anonymized movie ratings from their own database. Each record included an anonymized user ID, some information about the movie and the user's rating information. 

\subsection{Group Insurance Commision}

\subsection{New York Taxi dataset} 


\cleardoublepage