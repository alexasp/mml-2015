%===================================== CHAP 2 =================================

\chapter{Background and Motivation}

In this section we will first explain some basic concepts and expressions that are used in the privacy context such anonymization operations and Personally Identifiable Information(PII). Then we will have a look at some classic examples of failure to preserve privacy when data publishing and how these attacks motivated us to choose our topic for this thesis.

\section{Concepts and Expressions}

\section{Privacy Breaches}
In the recent years there have been many failures in privacy preserving data publishing. Many companies have been faced with a PR disaster after releasing data about their customers thinking them being anonymized, only to have people de-anonymize their data and breaching the privacy of the datasets' participants. In this section we will have a look at some of these privacy failures.

\subsection{Netflix Prize Competition}
Netflix, the world's largest online movie streaming website, decided in 2006 to crowdsource a new movie suggestion algorithm and offered a cash prize of 1 million dollar for the most efficient algorithm. To help the research, they released 100 million supposedly anonymized movie ratings from their own database. Each record included an anonymized user ID, some information about the movie and the user's rating information. Two researchers from the University of Texas demonstrated that an adversary who knows only a little bit about an individual subscriber can easily identify this subscriber's record in the dataset \cite{narayanan2006netflix}. Using the publicly available dataset from the Internet Movie Database (IMDB) as the source of background knowledge, they matched certain subscribers with their Netflix records, and uncovered their apparent political preferences and other potentially sensitive information\cite{narayanan2008robust}.

\subsection{Group Insurance Commission}
In 1997, Latanya Sweeney wrote a paper on how she had identified the medical records of Massachussets governor William Weld based on publicly available information from the database of Group Insurance Commission. She achieved this analyzing data from a public voter list, and linked it with patient-specific medical data through a combination of birth date, zip code, and gender\cite{sweeney2002k}. As these columns were similar in both databases, their combination could be used to identify medical records that belong to either one person, or a small group of people. Sweeney hypothesized that 87\% of the US population could be identified by having the combination of the three aforementioned records. It's worth noting here that this theory is not conclusive. A paper by Daniel Barth-Jones suggests that the re-identification of Weld may have been a fluke due to his public figure, and that ordinary people risk of identification is much lower\cite{barth2012re}. 

\subsection{New York Taxi dataset} 
The New York City Taxi and Limousine Commission released a dataset in 2013 containing details about every taxi ride that year, including pickup and dropoff times, location, fare, as well as anonymized (hashed) versions of the taxi's license and medallion numbers. Vijay Pandurangan, a researcher for Google, wrote a blog-post where he showed how he exploited a vulnerability in the hashing-function to re-identify the drivers. He then showed how this could be potentially used to calculate any driver's personal income\cite{vijay2014online}. 

Another researcher, called Anthony Tockar, wrote an article during his internship at Neustar Research where he proved that the dataset also contained an inherent privacy risk to the passengers which had been riding New York Taxis. Even though there was no information in the dataset on who had been riding the taxis, Tockar showed that by using auxiliary information such as timestamped pictures, he could stalk celebrities and figure out to where they were driving, and how much they tipped the driver. He also used map data from Google Maps to create a map of dropoff locations for people that had exited a late night visit from gentleman's club and taken a cab home. He then used websites like Spokeo and Facebook to find the cab customer's ethnicity, relationship status, court records, and even a profile picture\cite{tockar2014online}.	

\cleardoublepage