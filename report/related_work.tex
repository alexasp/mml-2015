
\section{Related Work}

Boutet et al. worked on a privacy-preserving distributed collaborative filtering which relied on user profile obfuscation and randomized response.\cite{boutet2013DisCollFil}.

Boutsis et al developed a participatory sensing system for smartphones which assumed that the data was distributed locally. Their system was called LOCATE, which handled ensured the privacy of the users, but they did not provide a differential privacy guarantee. \cite{boutsis2013}

Chaudhuri and Monteloni designed a logistic regression algorithm which guaranteed differential privacy in 2009, but their algorithm was designed to run on a single centralized database\cite{chaudhuri2009logistic}.

Han et al. investigated the problem of preserving differential privacy in distributed constrained optimization. By creating an algorithm based on stochastic gradient descent, which preserved privacy by adding noise to the public coordination signals (i.e gradients). \cite{han2014disOptimization}

Ji et al. \cite{ji2014DisLogReg} recently proposed a distributed solution using logistic regression, which learned from both private and publicly available medical datasets. Their solution differ from our own as they employ a globally synchronized structure, whereas our own solution works asynchronously. Their approach also requires a global aggregation of gradients, compared to our own which employ distributed ensemble learning,

Pathak et al \cite{pathak2010diffprivhomo} proposed a privacy-preserving protocol for composing a differentially private aggregate classifier. Their protocol trained classifiers locally in different parties, and the parties would then interact with an curator through a homomorphic encryption scheme to create a perturbed aggregate classifier. We took inspiration from their protocol when we created our own ensemble classifier.
\todo[inline]{Look at the last line and see if it's correct.}  

\todo{Add a reference to the multiparty gradient descent paper that cited Pathak, and somewhere rationalize why we chose the Pathak approach despite worse theoretical }


